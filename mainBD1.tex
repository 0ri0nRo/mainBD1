\documentclass{article}
\usepackage{graphicx}
\usepackage[a4paper, left=2cm, right=2cm, top=2.5cm, bottom=2cm]{geometry}
\usepackage[small]{titlesec}
\usepackage{subcaption} 
\usepackage[italian]{babel}
\usepackage[hidelinks]{hyperref}
\usepackage{listings}
\usepackage{sectsty}
%\usepackage[light]{CormorantGaramond}
\usepackage{fancyhdr}
\usepackage{footnote}
\usepackage{tgadventor}
\usepackage{titling}
\usepackage{graphicx}
\usepackage{tcolorbox}
\usepackage{multirow}
\usepackage{booktabs}
\usepackage{makecell}
\usepackage{algpseudocode}
\usepackage{algorithm}
\usepackage{mathtools}

\setlength{\droptitle}{-5em}   % regola la posizione del titolo
\pretitle{\begin{center}\LARGE}   % imposta la dimensione della font del titolo
\posttitle{\end{center}}
%\lstset{language=C}
\renewcommand\bfdefault{bx}
\sectionfont{\fontsize{20.74}{15}\selectfont}
\subsectionfont{\fontsize{17.28}{15}\selectfont}
\subsubsectionfont{\fontsize{12}{15}\selectfont}
\title{}

\author{}
\date{\today}


\begin{document}
\pagestyle{fancy}
\fancyhf{}
%\rhead{\thepage}
%\lhead{\rightmark}
\rfoot{\thepage}
\lhead{\quad \leftmark}
\rhead{ \quad \rightmark}

\renewcommand{\headrulewidth}{0.2pt}



\begin{titlepage}
    \begin{figure}[t]
        \centering
        \includegraphics[width=0.7\textwidth]{im/logo_sapienza_new.png}
        \label{fig:logo}
    \end{figure}    
    \null\vfill
    \begin{center}
      {\Huge Appunti di Basi Dati Modulo I} \\[2cm]
      {\Large Colacel Alexandru Andrei}
    \end{center}
    \vfill\null
    \renewcommand{\abstractname}{Disclaimer}

    
    
    \begin{abstract}  
    
    \hrulefill


    Le fonti sono le "Hand Notes" del prof. Perelli tradotte in italiano, appunti presi dalle slides della prof. De Marsico ed eventuali e-mail.\\
    \textbf{Nota: è vietata assolutamente la vendita di questo materiale in qualsiasi forma senza il mio consenso.} 
    \hrulefill
    \end{abstract}
  \end{titlepage}


\pagebreak
\tableofcontents

\pagebreak


\section{Lemma della Chiusura}
Sia $R$ uno schema e sia $F$ un insieme di dipendenze funzionali definite su $R$. Si ha che:\par 
\begin{equation}
X \rightarrow Y \in F^{A} \Longleftrightarrow Y \subseteq X^{+}
\end{equation}

\subsection{Dimostrazione $\Rightarrow$}

Dato $X$ $\rightarrow$ $Y \in F^{A}$, per la regola della decomposizione, otteniamo:
\begin{equation}
X \rightarrow A \in F^{A}, \quad \forall A \in Y  
\end{equation} 
e quindi, per definizione di $X^{+}$, otteniamo che: 
\begin{equation}
A \in X^{+}, \quad \forall A \in Y  
\end{equation}
che significa: 
\begin{equation}
Y \subseteq X^{+}
\end{equation}


\subsection{Dimostrazione $\Leftarrow$}
Dato: 
\begin{equation}
Y \subseteq X^{+}
\end{equation}
si ottiene che: 
\begin{equation}
X \rightarrow A \in F^{A} \quad \forall A \in Y
\end{equation}
che implica, per la regola dell'unione, che: 
\begin{equation}
X \rightarrow Y \in F^{A}
\end{equation}

\pagebreak
\section{Teorema $F^{+}$ = $F^{A}$}

Dato uno schema $R$ e un insieme $F$ di dipendenze funzionali definite su $R$, si ha che:
\begin{equation}
  F^{+} = F^{A}
\end{equation}

\subsection{Dimostrazione $F^{A} \subseteq F^{+}$}
Prendiamo $X \rightarrow Y \in F^{A}$, noi dobbiamo provare che $X \rightarrow Y \in F^{+}$ per induzione con $n$ numero di applicazioni degli assiomi di Armstrong.
\begin{itemize}
  \item \textbf{Caso base} (n = 0): se $X \rightarrow Y \in F^{A}$ senza aver applicato alcun assioma di Armstrong, allora l'unica possibilità è che:
  \begin{equation}
    X \rightarrow Y \in F \subseteq F^{+}
  \end{equation}
  \item \textbf{Ipotesi induttiva forte:} ogni dipendenza funzionale in $F^{A}$ ottenuta da $F$
  applicando k $\leq$ n assiomi di Armstrong è anche in $F^{+}$:
  \begin{center}
    $X \rightarrow Y \in F^{A}$ tramite $k \leq n$ assiomi $\Rightarrow X \rightarrow Y \in F^{+}$
  \end{center}
  \item \textbf{Passo induttivo:} è necessario dimostrare che se $X \rightarrow Y \in F^{A}$ dopo aver applicato $n +1$ assiomi di Armstrong, allora $ X \rightarrow Y \in F^{+}$.\par
  È possibile ritrovarsi in uno dei seguenti tre casi:
  \begin{enumerate}
    \item Se l'$(n + 1)$-esimo assioma applicato è l'assioma di \textbf{riflessività}, allora l'unica possibilità è che:
    \begin{equation}
      X \rightarrow Y \in F^{A} \Leftrightarrow Y \subseteq X \subseteq R
    \end{equation}
    Dunque, poiché, $Y \subseteq X \subseteq R$, per ogni istanza legale di $R$ si ha che:
    \begin{equation}
      \forall t_1, t_2 \in r_{1}, t_1[X] = t_2[X] \Rightarrow t_1[Y] = t_2[Y]
    \end{equation}
    da cui ne segue automaticamente che $X \rightarrow Y \in F^{+}$
    \item Se l'$(n + 1)$-esimo assioma applicato è l'assioma di \textbf{aumento}, allora è obbligatoriamente necessario che:
    \begin{itemize}
      \item $\exists V, W \subseteq R \, | \, \exists V \rightarrow W \in F_{A}$, ottenuta applicando $j \leq n$ assiomi di Armstrong\\

      \item $\exists Z \subseteq R \, | \, X := VZ, \, Y := WZ$\\
    \end{itemize}
      Affinché si abbia che:
      \begin{equation}
        Z \subseteq R, \, V \rightarrow W \Rightarrow VZ \rightarrow WZ = X \rightarrow Y \in F^{A}
      \end{equation}

      Siccome per ipotesi induttiva si ha $V \rightarrow W \in F^{A} \Rightarrow V \rightarrow W \in F^{+}$ e siccome $Z \subseteq Z \Rightarrow Z \rightarrow Z \in F^{+}$, si vede facilmente che: 
      \begin{figure}[hbt]
        \begin{center}
            \includegraphics[width=0.8\textwidth,keepaspectratio]{{im/1}}
        \end{center}
        \end{figure}
        \pagebreak
      \item Se l'$(n + 1)$-esimo assioma applicato è l'assioma di \textbf{transitività}, allora è obbligatoriamente necessario che $\exists X \rightarrow Z, Z \rightarrow Y \in F^{A}$, ottenute con $k \leq n$ assiomi di Armstrong, affinché si abbia che:
      \begin{equation}
        X \rightarrow Z \in F^{A} \lor Z \rightarrow Y \in F^{A} \Rightarrow X \rightarrow Y \in F^{A}
      \end{equation}
      Siccome per ipotesi induttiva $X \rightarrow Z \in F^{A} \Rightarrow X \rightarrow Z \in F^{+}$ e $Z \rightarrow Y \in F^{A} \Rightarrow Z \rightarrow Y \in F^{+}$, si vede facilmente che:
      \begin{figure}[hbt]
        \begin{center}
            \includegraphics[width=0.7\textwidth,keepaspectratio]{{im/2}}
        \end{center}
        \end{figure}
  \end{enumerate}
\end{itemize}

\pagebreak
\subsection{Dimostrazione $F^{+} \subseteq F^{A}$}
\begin{figure}[hbt]
  \begin{center}
      \includegraphics[width=0.75\textwidth,keepaspectratio]{{im/3}}
  \end{center}
  \end{figure}
  \begin{figure}[hbt]
    \begin{center}
        \includegraphics[width=0.8\textwidth,keepaspectratio]{{im/4}}
    \end{center}
    \end{figure}
    
    \begin{tcolorbox}[colback=white!20!white,colframe=green!70!black, title=Nota]
      Poiché $F^{+} = F^{A}$, per calcolare $F^{+}$ ci basta applicare gli assiomi di Armstrong sulle dipendenze in $F$ in modo da trovare $F^{A}$. \par Tuttavia, calcolare $F^{+} = F^{A}$ richiede tempo esponenziale, quindi $O(2^{nk})$: considerando anche solo l'assioma di riflessività, siccome ogni possibile sottoinsieme di $R$ genera una dipendenza e siccome i sottoinsiemi possibili di $R$ sono $2^{|R|}$, allora ne segue che $|F^{+}| >> 2^{|R|}$.
    \end{tcolorbox}

    




\pagebreak
\section{Chiusura di X}
\texttt{Input}:
\begin{itemize}
  \item Relazione $R$
  \item Dipendenze Funzionali $F$
  \item Insieme $X \subseteq R$
\end{itemize}
\texttt{Output}: $X^{+}$\\

\begin{algorithm}
  \caption{Closure Algorithm}  
    \begin{algorithmic}[1]
  \State $Z \gets X$
  \State $S \gets \{A \mid \exists Y \rightarrow V \in F, A \in V \land Y \subseteq Z\}$
  \While{$S \not\subseteq Z$}
    \State $Z \gets Z \cup S$
    \State $S \gets \{A \mid \exists Y \rightarrow V \in F, A \in V \land Y \subseteq Z\}$
  \EndWhile
  \State \textbf{return} $Z$
  \end{algorithmic}
\end{algorithm}
Tale algoritmo viene eseguito in tempo polinomiale, ossia $On^{k}$

\subsection{Teorema: L'algoritmo computa $X_{F}^{+}$}
Il teorema calcola correttamente la chiusura di un insieme di attributi X rispetto ad un insieme F di dipendenze funzionali.

\textbf{Dimostrazione}:\par
Denotiamo:\par
\begin{equation}
  Z_{0}, Z_{1}, \dots, Z_{i}, \dots 
\end{equation}
\begin{equation}
  S_{0}, S_{1}, \dots, S_{i}, \dots
\end{equation}
Indichiamo con $Z_{0}$ il valore iniiale di $Z (Z_{0} = X)$ e con $Z_{i}$ e $S_{i}$, $i \geq 1$ i valori di $Z$ ed $S$ dopo l'$i$-esima esecuzione del corpo del ciclo, infatti notiamo che $Z_{i} \subseteq Z_{i+1}$, per ogni $i$.\par
Sia $j$ tale che $S_{j} \subseteq Z_{j}$ (cioè,  $Z_{j}$ è l'output di $Z$ quando l'algoritmo termina); proveremo che:
\begin{equation}
  A \in Z_{j} \Leftrightarrow A \in X^{+} 
\end{equation}
\begin{itemize}
  \item Dimostriamo per induzione su $i$ che $Z_{i} \subseteq X^{+}$, per ogni $i$
  \begin{itemize}
    \item Caso base dell'induzione: i = 0.\par Poiché  $Z_{0} = X$ e $X \subseteq X^{+}$ si ha $Z_{0} \subseteq X^{+}$
    
    \item Induzione: $i > 0$.\par Per l'ipotesi induttiva $Z_{i-1} \subseteq X_{+}$. \par Sia A un attributo in $Z_{i} - Z_{i-1}$ deve esistere una dipendenza $Y \rightarrow V \in F$ tale che $Y \subseteq Z_{i-1}$ e $A \in V$. Poiché $Y \subseteq Z_{i-1}$ per l'ipotesi induttiva si ha che $Y \subseteq X_{+}$ pertanto per il lemma $X \rightarrow Y \in F^{A}$.\par Poiché $X \rightarrow Y \in F^{A}$ e $Y \rightarrow V \in F$ per l'assioma della tansitività si ha $X \rightarrow V \in F^{A}$ e quindi per il lemma, $V \subseteq X^{+}$. Pertanto per ogni $A \in Z_{i} - Z_{i-1}$ si ha $A \in X^{+}$. Da ciò segue per ipotesi induttiva che $Z_{i} \subseteq X^{+}$
  \end{itemize}
  \pagebreak
  \item Dimostriamo ora che $X^{+} \subseteq Z_{i}$:\par
  \begin{itemize}
    \item Sia $X \subseteq R$ e sia $r$ istanza di $R(Z_{i}, R-Z_{i})$. 
    \begin{table}[ht]
      \centering
      \begin{tabular}{|c|c|c|c|c|c|}
      \hline
      \multicolumn{3}{|c|}{$Z_{i}$} & \multicolumn{3}{c|}{$R - Z_{i}$} \\
      \hline
      $A_1$ & $\dots$ & $A_{i}$ & $A_{j}$ & $\dots$  & $A_{n}$ \\
      \hline
      $1$ & $\dots$ & $1$ & $1$ & $\dots$  & $1$ \\
      \hline
      $1$ & $\dots$ & $1$ & $0$ & $\dots$  & $0$ \\
      \hline
      \end{tabular}
  \end{table}

  dunque tale che per $t_{1}, t_{2} \in r$ si ha:
  \begin{itemize}
    \item $t_{1}[Z_{i}] = (1, \dots, 1) = t_{2}[Z_{i}]$
    \item $t_{1}[R - Z_{i}] = (1, \dots, 1) \neq (0, \dots, 0) = t_{2}[R - Z_{i}]$
  \end{itemize}
    \item Notiamo che $\forall V, W \subseteq R$ | $V \rightarrow W \in F$ si ha che:
    \begin{itemize}
      \item Se $V \cap (R - Z_i) \neq \emptyset$ (quindi anche se $V \subseteq R - Z_i$) allora $t1[V] \neq t2[V]$, quindi $r$ soddisfa $V \rightarrow W \in F$
      \item Se invece $V \subseteq Z_i$, allora $W \subseteq S_f$, poiché per come viene calcolato $S_f$, si ha che:\par
      $V \rightarrow W \in F$, $V \subseteq Z_i$, $B \in W \subseteq R \Rightarrow B \in S_f \Rightarrow W \subseteq S_f$\par
      e quindi, siccome $S_f \subseteq Z_i$ è la condizione che termina l'algoritmo, allora $W \subseteq S_f \subseteq Z_i$
      \item Siccome $V, W \subseteq Z_i$, in definitiva si ha che\par $t1, t2 \in r$, $t1[V] = (1, \dots, 1) = t2[V]$ e $t1[W] = (1, \dots, 1) = t2[W]$,\par e quindi $r$ soddisfa $V \rightarrow W \in F$
    \end{itemize}
    \item Siccome in entrambi i casi $r$ soddisfa ogni $V \rightarrow W \in F$, allora $r$ è legale.
    \item A questo punto, dato $A \in X^+$, si ha che $X \rightarrow A \in F^A = F^+$ deve essere soddisfatta da qualsiasi istanza legale di $R$, inclusa $r$ stessa.
    \item Poiché $X = Z_0 \subseteq Z_i$, ne segue che la dipendenza non può essere soddisfatta a vuoto poiché $t1[X] = t2[X]$. Dunque, l'unica possibilità affinché $X \rightarrow A \in F^+$ sia soddisfatta da $r$ è $A \in Z_i$ in modo che si abbia $t1[A] = t2[A]$.
    \item Dunque, siccome $A \in X^+ \Rightarrow A \in Z_i$, concludiamo che $X^+ \subseteq Z_i$.
  
  \end{itemize}
\end{itemize}

\pagebreak
\section{Lemma: Inclusione delle chiusure}
Dato uno schema $R$ e due insiemi $F$ e $G$ di dipendenze funzionali su $R$, si ha che:\par 
\begin{equation}
F \subseteq G^+ \Leftrightarrow F^+ \subseteq G^+
\end{equation}
\subsection{Dimostrazione}
\begin{itemize}
  \item Denotiamo come $G \xrightarrow{\text{A}} F$ la possibilità di ottenere F partendo da G applicando una determinata quantità di assiomi di Armstrong.
  \item  Ricordando che $G^A$ è l'insieme di tutte le dipendenze funzionali ottenibile applicando assiomi di Armstrong su $G$, allora:\par
  \begin{equation}
    G \xrightarrow{\text{A}} F \Leftrightarrow \forall X \rightarrow Y \in F, \text{ si ha } X \rightarrow Y \in G^A = G^+ \Leftrightarrow F \subseteq G^+
  \end{equation}
  \item Siccome $F \subseteq G \Leftrightarrow G \xrightarrow{\text{A}} F$, per definizione di $F^A = F^+$ si ha che:
  \begin{equation}
    F \subseteq G^+ \Rightarrow G \xrightarrow{\text{A}} F \xrightarrow{\text{A}} F^A = F^+ \Rightarrow F^+ \subseteq G^+
  \end{equation}
  \item Viceversa, si ha che $F^+ \subseteq G^+ \Rightarrow F \subseteq F^+ \subseteq G^+$, quindi concludiamo che $F \subseteq G^+ \Leftrightarrow F^+ \subseteq G^+$
\end{itemize}







\pagebreak
\section{Chiusura di $X$ in $G$}
\pagebreak
\section{Join senza perdita}
\pagebreak
\section{Assiomi di Armstrong}
\pagebreak
\section{Decomposizione che preserva $F$}
\subsection{Prima proprietà}
\subsection{Seconda proprietà}
\subsection{Terza proprietà}
\subsection{Definizione $G$}
\pagebreak
\section{Hash}
\pagebreak
\section{Dimostrazione $\rho$ preserva $F$}
\pagebreak
\section{B-tree}
\pagebreak
\section{Chiusura di un insieme di attributi}
\subsection{Algoritmo}
\subsection{Dimostrazione correttezza}
\pagebreak
\section{3NF (per dipendenza transitiva)}
\pagebreak
\section{Chiusura di F e primo lemma}
\pagebreak
\section{Isam}
\subsection{Variante Isam con chiavi indice che hanno valore ultimo record}
\pagebreak
\section{Altre definizioni da sapere}
\subsection{Chiave minimale} 
\subsection{Superchiave}


\end{document}