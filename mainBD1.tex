\documentclass{article}
\usepackage{graphicx}
\usepackage[a4paper, left=2cm, right=2cm, top=2.5cm, bottom=2cm]{geometry}
\usepackage[small]{titlesec}
\usepackage{subcaption} 
\usepackage[italian]{babel}
\usepackage[hidelinks]{hyperref}
\usepackage{listings}
\usepackage{sectsty}
%\usepackage[light]{CormorantGaramond}
\usepackage{fancyhdr}
\usepackage{footnote}
\usepackage{tgadventor}
\usepackage{titling}
\usepackage{graphicx}
\usepackage{tcolorbox}
\usepackage{multirow}
\usepackage{booktabs}
\usepackage{makecell}

\setlength{\droptitle}{-5em}   % regola la posizione del titolo
\pretitle{\begin{center}\LARGE}   % imposta la dimensione della font del titolo
\posttitle{\end{center}}
%\lstset{language=C}
\renewcommand\bfdefault{bx}
\sectionfont{\fontsize{20.74}{15}\selectfont}
\subsectionfont{\fontsize{17.28}{15}\selectfont}
\subsubsectionfont{\fontsize{12}{15}\selectfont}
\title{}

\author{}
\date{\today}


\begin{document}
\pagestyle{fancy}
\fancyhf{}
%\rhead{\thepage}
%\lhead{\rightmark}
\rfoot{\thepage}
\lhead{\quad \leftmark}
\rhead{ \quad \rightmark}

\renewcommand{\headrulewidth}{0.2pt}



\begin{titlepage}
    \begin{figure}[t]
        \centering
        \includegraphics[width=0.7\textwidth]{im/logo_sapienza_new.png}
        \label{fig:logo}
    \end{figure}    
    \null\vfill
    \begin{center}
      {\Huge Appunti di Basi Dati Modulo I} \\[2cm]
      {\Large Colacel Alexandru Andrei}
    \end{center}
    \vfill\null
    \renewcommand{\abstractname}{Disclaimer}

    
    
    \begin{abstract}  
    
    \hrulefill


    Le fonti sono le Hand Notes del prof tradotte in italiano con l'obiettivo di migliorare la leggibilità.\\
    \textbf{Nota: è vietata assolutamente la vendita di questo materiale in qualsiasi forma senza il mio consenso.}  

    
    \hrulefill

    \end{abstract}
  \end{titlepage}


\pagebreak
\tableofcontents

\pagebreak


\section{Lemma della Chiusura}
Sia $R$ uno schema e sia $F$ un insieme di dipendenze funzionali definite su $R$. Si ha che:\par 
\begin{equation}
X \rightarrow Y \in F^{A} \Longleftrightarrow Y \subseteq X^{+}
\end{equation}

\subsection{Dimostrazione $\Rightarrow$}

Dato $X$ $\rightarrow$ $Y \in F^{A}$, per la regola della decomposizione, otteniamo:
\begin{equation}
X \rightarrow A \in F^{A}, \quad \forall A \in Y  
\end{equation} 
e quindi, per definizione di $X^{+}$, otteniamo che: 
\begin{equation}
A \in X^{+}, \quad \forall A \in Y  
\end{equation}
che significa: 
\begin{equation}
Y \subseteq X^{+}
\end{equation}


\subsection{Dimostrazione $\Leftarrow$}
Dato: 
\begin{equation}
Y \subseteq X^{+}
\end{equation}
si ottiene che: 
\begin{equation}
Y \subseteq A \in F^{A} \quad \forall A \in Y
\end{equation}
che implica, per la regola dell'unione, che: 
\begin{equation}
X \rightarrow Y \in F^{A}
\end{equation}

\pagebreak
\section{FA = F$^{+}$}
\pagebreak
\section{Chiusura di X}
\pagebreak
\section{Lemma Chiusura Inclusione}
\pagebreak
\section{Chiusura di X in G}
\pagebreak
\section{Join senza perdita}


\end{document}